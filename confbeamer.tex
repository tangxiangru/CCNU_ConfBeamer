%# -*- coding:utf-8 -*-
\documentclass[10pt,aspectratio=169,mathserif]{beamer}		
%设置为 Beamer 文档类型,设置字体为 10pt,长宽比为16:9,数学字体为 serif 风格


%%%%-----导入宏包-----%%%%
\usepackage{ccnu}			%导入 CCNU 模板宏包
\usepackage{ctex}			%导入 ctex 宏包,添加中文支持
\usepackage{amsmath,amsfonts,amssymb,bm}   %导入数学公式所需宏包
\usepackage{color}			 %字体颜色支持
\usepackage{graphicx,hyperref,url}
\usepackage{metalogo}	% 非必须
%% 上文引用的包可按实际情况自行增删
%%%%%%%%%%%%%%%%%%	
\usepackage{CJK}

\beamertemplateballitem		%设置 Beamer 主题

%%%%------------------------%%%%%
\catcode`\。=\active         %或者=13
\newcommand{。}{.}				
%将正文中的“。”号转换为“.”。中文标点国家规范建议科技文献中的句号用圆点替代
%%%%%%%%%%%%%%%%%%%%%

%%%%----首页信息设置----%%%%
\title[SEKE 2019]{Improving Code Generation From Descriptive Text By Combining Deep Learning
and Syntax Rules}
%\subtitle{——这里是副标题}			
%%%%----标题设置


\author[Xiangru Tang]{
  Xiangru Tang, Zhihao Wang, Jiyang Qi, Zengyang Li \\\medskip
  {\small \url{xrtang@mails.ccnu.edu.cn, zhihaowang@hust.edu.cn}} \\
  {\small \url{jyqi@hust.edu.cn, zengyangli@mail.ccnu.edu.cn}}}
%%%%----个人信息设置
  
\institute[IOPP]{
 School of Computer Science, Central China Normal University, Wuhan, China\\
Hubei Provincial Key Laboratory of Artificial Intelligence and Smart Learning, Central China Normal University, China\\
School of Computer Science and Technology, Huazhong University of Science and Technology, Wuhan, China}
%%%%----机构信息

\date[Thursday, July 11, 2019]{
  Thursday, July 11, 2019}
%%%%----日期信息
  
\begin{document}

\begin{frame}
\titlepage
\end{frame}				%生成标题页

\begin{frame}
\frametitle{Outline}
\tableofcontents
\end{frame}				%生成提纲页

\section{Problem \& Motivation}
\begin{frame}
  \frametitle{Introduction}

  \begin{itemize}
    \item {编译方式}
	    \begin{itemize}
	    	\item  推荐安装完整版的 TeXLive
	    	\item 使用 \XeLaTeX 编译
	    \end{itemize}
    \item 请参考 \LaTeX 和 Beamer 用户文档 
    
    \item 行内数学公式示例 $\sin^2 \theta + \cos^2 \theta = 1$
    \item {行间数学公式示例 \begin{equation}
	    y_{1}=\int \sin x\, {\rm d}x
    \end{equation}	 }   
    \item 基于“华大绿”颜色 \url{http://www.ccnu.edu.cn/}
  \end{itemize}
\end{frame}


\begin{frame}
  \frametitle{RELATED WORK}

  \begin{itemize}
    \item {编译方式}
	    \begin{itemize}
	    	\item  推荐安装完整版的 TeXLive
	    	\item 使用 \XeLaTeX 编译
	    \end{itemize}
    \item 请参考 \LaTeX 和 Beamer 用户文档 
    
    \item 行内数学公式示例 $\sin^2 \theta + \cos^2 \theta = 1$
    \item {行间数学公式示例 \begin{equation}
	    y_{1}=\int \sin x\, {\rm d}x
    \end{equation}	 }   
    \item 基于“华大绿”颜色 \url{http://www.ccnu.edu.cn/}
  \end{itemize}
\end{frame}




\section{Architecture}
\begin{frame}
  \frametitle{Task Definition}
	\begin{block}{Slides with \LaTeX}
	    Beamer offers a lot of functions to create nice slides using \LaTeX.
	  \end{block}
	
	  \begin{block}{The basis}
	    theme
	    \begin{itemize}
	      \item split
	      \item whale
	      \item rounded
	      \item orchid
	    \end{itemize}
	  \end{block}
\end{frame}

\begin{frame}
  \frametitle{CDS-POOLING}
	\begin{block}{Slides with \LaTeX}
	    Beamer offers a lot of functions to create nice slides using \LaTeX.
	  \end{block}
	
	  \begin{block}{The basis}
	    theme
	    \begin{itemize}
	      \item split
	      \item whale
	      \item rounded
	      \item orchid
	    \end{itemize}
	  \end{block}
\end{frame}


\begin{frame}
  \frametitle{CDS-CNN}
	\begin{block}{Slides with \LaTeX}
	    Beamer offers a lot of functions to create nice slides using \LaTeX.
	  \end{block}
	
	  \begin{block}{The basis}
	    theme
	    \begin{itemize}
	      \item split
	      \item whale
	      \item rounded
	      \item orchid
	    \end{itemize}
	  \end{block}
\end{frame}


\begin{frame}
  \frametitle{CDS-SAN}
	\begin{block}{Slides with \LaTeX}
	    Beamer offers a lot of functions to create nice slides using \LaTeX.
	  \end{block}
	
	  \begin{block}{The basis}
	    theme
	    \begin{itemize}
	      \item split
	      \item whale
	      \item rounded
	      \item orchid
	    \end{itemize}
	  \end{block}
\end{frame}


\section{Experiment}
\begin{frame}
  \frametitle{Research Questions}
	 \begin{enumerate}
	    \item This just shows the effect of the style
	    \item It is not a Beamer tutorial
	    \item Read the Beamer manual for more help
	    \item Contact me only concerning the style file
	  \end{enumerate}
\end{frame}


\begin{frame}
  \frametitle{Dataset}
	 \begin{enumerate}
	    \item This just shows the effect of the style
	    \item It is not a Beamer tutorial
	    \item Read the Beamer manual for more help
	    \item Contact me only concerning the style file
	  \end{enumerate}
\end{frame}


\begin{frame}
  \frametitle{Evaluation Metrics}
	 \begin{enumerate}
	    \item This just shows the effect of the style
	    \item It is not a Beamer tutorial
	    \item Read the Beamer manual for more help
	    \item Contact me only concerning the style file
	  \end{enumerate}
\end{frame}

\begin{frame}
  \frametitle{Experiment Hyperparameters}
	 \begin{enumerate}
	    \item This just shows the effect of the style
	    \item It is not a Beamer tutorial
	    \item Read the Beamer manual for more help
	    \item Contact me only concerning the style file
	  \end{enumerate}
\end{frame}

\begin{frame}
  \frametitle{Experimental results}
	 \begin{enumerate}
	    \item This just shows the effect of the style
	    \item It is not a Beamer tutorial
	    \item Read the Beamer manual for more help
	    \item Contact me only concerning the style file
	  \end{enumerate}
\end{frame}
\begin{frame}
  \frametitle{Case study}
	 \begin{enumerate}
	    \item This just shows the effect of the style
	    \item It is not a Beamer tutorial
	    \item Read the Beamer manual for more help
	    \item Contact me only concerning the style file
	  \end{enumerate}
\end{frame}



\section{Conclusion}
\begin{frame}
  \frametitle{Conclusion}

  \begin{itemize}
    \item Easy to use
    \item Good results
  \end{itemize}
\end{frame}

\begin{frame}
  \frametitle{Thanks}

  \begin{itemize}
    \item This work is partially supported by the National Natural Science Foundation of China (NSFC) under the Grant Nos. 61702377 and 61773175.
    \item Finally, \textbf{THANK YOU} for attending! We hope that you’ve learned something today!
  \end{itemize}
\end{frame}



\begin{frame}{REFERENCES}
\begin{thebibliography}{99} 
\bibitem{zhao1} Yi~Zhao, {\sl An introduction to X}, Sep.~15, 2015
\bibitem{qian2} Er~Qian, San~Sun, 
Phys.\ Lett.\ A {\bf xx}, 2xx (20xx)   
\bibitem{li4} Si~Li, Phys.\ Rev.\ C {\bf xx}, 5xx (20xx) 

\end{thebibliography}
\end{frame}


\end{document}